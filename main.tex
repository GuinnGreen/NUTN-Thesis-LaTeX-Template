\documentclass[
  degree    = master,                       % degree   = master  | doctor
  language  = chinese,                      % language = chinese | english
  fontset   = template,                     % fontset  = default | template
%   draft                                   % LaTex draft mode (images will change to box)
]{nutnthesis}

% !!!!!!!!!!!!!!!!!!!!!!!!!!!!!!!!!!!!!!!!!!!!!!!!!!!!!!!!!!!!!!!!!!!!
% !  注意:論文標題、作者、指導教授等資訊,請至 nutnsetup.tex 修改      !
% !!!!!!!!!!!!!!!!!!!!!!!!!!!!!!!!!!!!!!!!!!!!!!!!!!!!!!!!!!!!!!!!!!!!
%
% !TeX root = ./main.tex

% --------------------------------------------------
% 資訊設定(Information Configs)
% --------------------------------------------------

\nutnsetup{
  university*   = {National University of Tainan},
  university    = {國立臺南大學},
  college       = {教育學院},
  college*      = {College of Education},
  institute     = {教育學系},
  institute*    = {Department of Education},
  program       = {教育經營與管理碩士班},                 % 學程/班別(若無可留空)
  program*      = {Master's Program of Educational Entrepreneurship and Management},
  title         = {論文標題},
  title*        = {Thesis Title},
  author        = {王大明},
  author*       = {Yu-Sheng Cheng},
  advisor       = {陳教授},
  advisor*      = {Professor Chen},
  date          = {一百一十五年~二月},                    % ~ 表示空格
  date*         = {February~2026},
  keywords      = {關鍵字一, 關鍵字二, 關鍵字三},
  keywords*     = {Keyword1, Keyword2, Keyword3},
}

% --------------------------------------------------
% 加載套件(Include Packages)
% --------------------------------------------------

% 下列產生亂字的pkg可刪除
\usepackage{lipsum}                     % 英文亂字
\usepackage{zhlipsum}                   % 中文亂字


\begin{document}
	% 加入浮水印 Watermark
	% 浮水印須出現在封面和論文本文第一頁
	%
	\makewatermark{watermark}                         % 加入浮水印(Watermark)

	% 封面 Cover(附件一)
	% 若需要在封面加上(初稿)字樣,請在\makecover後加上{draft}
	%
	% \makecover{draft}                               % 有(初稿)字樣的論文封面
	\makecover                                        % 論文封面(Cover)

	% 書名頁 Title Page(附件二)
	\maketitlepage                                    % 書名頁(Title Page)

	% 口試審定 Verification Letter(附件四)
	% \makeverification會生成空白的審定書
	% 若有審定書pdf檔案,可用\renderverification{檔案路徑}渲染審定書
	%
	\makeverification                                 % 口試委員審定書(Verification Letter)
	% \renderverification{frontpages/verification}    % 渲染口試委員審定書(Render Verification Letter)

	% 中英文摘要 Abstract(附件六、附件七)
	\frontmatter
	\input{frontpages/abstract}                       % 摘要(Abstract)

	% 謝辭 Acknowledgement(附件十六)
	% 台南大學規範:謝辭置於摘要之後
	\input{frontpages/acknowledgement}                % 謝辭(Acknowledgement)

	% 生成目錄與符號列表 Tables of Contents and Denotation
	\maketableofcontents                              % 目次(Table of Contents)
	\makelistoftables                                 % 表次(List of Tables)
	\makelistoffigures                                % 圖次(List of Figures)
	\input{frontpages/denotation}                     % 符號說明(Denotation)

	% 論文內容 Contents of Thesis
	\mainmatter
	\input{sections/01introduction}                   % 緒論(Introduction)
	\input{sections/02related_work}                   % 文獻探討(Related Work)
	\input{sections/03method}                         % 研究方法(Method)
	\input{sections/04experiments}                    % 研究結果(Experiments)
	\input{sections/05conclusion}                     % 結論(Conclusion)

	% 參考文獻 References
	\refmatter
	\bibliographystyle{apalike}                       % 參考文獻格式:APA(教育學院)
	\bibliography{backpages/references}               % 參考文獻資料庫(References Database)

	% 附錄 Appendix
	\input{backpages/appendix}                        % 附錄(Appendix)
\end{document}
